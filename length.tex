\documentclass{article}

\usepackage{amsmath}


\begin{document}
\title{Equations of Tubing Length}
\author{Lake Hu}
\maketitle

\section{Introduction}
This document describes the equations of tubing length, where the tubing is
hanged in a well is full of some kind of fluids, such as water, oil, or
mud. Many factors in a well make the tubing longer or shorter because of
force, temperature.

\section{Tubing Sealed in Packer}
\subsection{Neutral Point}
The neutral point is a point of which the forces, include axial, radial,
and tangential, are balance.

\begin{equation} \label{eqNeutral}
n = \frac{F}{w}
\end{equation}
\begin{center}
\begin{tabular}{c l}
$F$ & Force, compression is positive. \\
$w$ & weight of tubing per unit. \\
\end{tabular}
\end{center}

\subsubsection{Comments}
$F$ is actual existing, \emph{not fictitious}. $w$ is calculated by Eq. \ref{eqWeight} if there are fluids.
\begin{equation} \label{eqWeight}
w = w_s + w_i - w_o
\end{equation}
\begin{center}
\begin{tabular}{c l}
$w_s$ & average weight of tubing per unit length. \\
$w_i$ & weight of liquid in the tubing per unit length. \\
$w_o$ & weight of outside liquid displaced per unit length. \\
\end{tabular}
\end{center}

\subsection{Hooke's Law}

\begin{equation} \label{eqHooke}
\Delta L_1 = - \frac{LF}{EA_s}
\end{equation}
\begin{center}
\begin{tabular}{c l}
$F$ & Force, compression is positive. \\
$L$ & Length of tubing. \\
$E$ & Young's modulus \\
$A_s$ & cross sectional area of tubing wall \\
\end{tabular}
\end{center}

\subsubsection{Comments}
$F$ is actually existing (\emph{nonfictitious}). The paper gives a equation of
packer existing. I wonder what's the equation if no packer.
\begin{equation}\label{eqActualForce}
F_a = (A_p - A_i)P_i - (A_p - A_o)P_0
\end{equation}
\begin{center}
\begin{tabular}{c l}
$A_p$ & Area corresponding to packer bore ID \\
$A_o$ & Area corresponding to tubing OD \\
$A_i$ & Area corresponding to tubing ID \\
$P_i$ & Pressure inside the tubing at the packer level \\
$P_o$ & Pressure outside the tubing at the packer level \\
\end{tabular}
\end{center}
Without packer, the $P_i$ is equal to the $P_o$, and the $A_p$ is equal to the $P_o$.


\subsection{Helical Buckling Length}

\begin{equation} \label{eqBuckle1}
\Delta L_2 = - \frac{r^2F^2}{8EIw} 
\end{equation}
\begin{equation} \label{eqBuckle2}
\Delta L_2 = - \frac{r^2F^2}{8EIw}\left(\frac{Lw}{F}\left(2-\frac{Lw}{F}\right)\right)
\end{equation}

\begin{center}
\begin{tabular}{c l}
$r$ & tubing-to-casing radial clearance \\
$F$ & fictitious force \\
$E$ & young's modulus \\
$I$ & moment of inertia of tubing cross-section. see Eq. \ref{eqInertia}. \\
$w$ & weight of tubing per unit length. see Eq. \ref{eqWeight}. \\
\end{tabular}
\end{center}

\begin{equation} \label{eqInertia}
I=\frac{\pi}{64}(D^4 - d^4)
\end{equation}
\begin{center}
\begin{tabular}{c l}
$D$ & out diameter of tubing. \\
$d$ & inner diameter of tubing. \\
\end{tabular}
\end{center}

\subsubsection{Comments}
The first equation is for the length change resulting from helical buckling
when the neutral point is within the string. When the calculated value of the
neutral point is above the upper end of the string, the second equation is
used.

The force $F$ is called fictitious because the actual force is too complex
when tubing is buckling. 
\begin{equation}\label{eqFictitiousForce}
F = A_p(P_i - P_o)
\end{equation}
 Without packer, the $P_i$ is equal to $P_o$ and the $A_p$ is $A_o$. The $F$
 is zero.

We can get a equation below by combining Eq. \ref{eqActualForce} and
Eq. \ref{eqFictitiousForce}.
\begin{equation}\label{eqA2F}
F_f = F_a + A_iP_i - A_oP_o
\end{equation}
where $F_f$ is the force in Eq. \ref{eqFictitiousForce} and $F_a$ is the force
in Eq. \ref{eqActualForce}.

The weight $w$ of tubing in the presence of liquids should be $w = w_s + w_i -
w_o$, where $w_s$ is the average (i.e., including couplings) weight of tubing
per unit length, $w_i$ is the weight of liquid in the tubing per unit length
and $w_o$ is the weight of outside liquid displaced per unit length.

\subsection{Due to fluid}
\begin{equation}
\Delta L_3 = - \frac{\nu}{E}\frac{\Delta \rho_i - R^2\Delta \rho_o -
  \frac{1+2\nu}{2\nu}\delta}{R^2 - 1}L^2 - \frac{2\nu}{E}\frac{\Delta p_i -
  R^2\Delta p_o}{R^2 - 1}L
\end{equation}
\begin{center}
\begin{tabular}{c l}
$\delta$ & drop of pressure in the tubing due to flow per unit length.\\
$\nu$ & Poisson's ratio of the material (for steel, it's 0.3). \\
$E$ & young's modulus \\
$\Delta \rho_i$ & change in density of liquid in the tubing. \\
$\Delta \rho_o$ & change in density of liquid in the annulus. \\
$R$ & ratio $OD/ID$ of the tubing. \\
$\Delta p_i$ & change in surface tubing pressure. \\
$\Delta p_o$ & change in surface annulus pressure. \\
\end{tabular}
\end{center}

\subsubsection{Comments}
$\delta$ is the drop of pressure. Positive when the flow is downward. The
\emph{drop} means lower, doesn't it? If so, what's about the higher? And
what's about the change of pressure due to the explosion. I think the flow
means the whole tubing's, not a part.

Meanwhile, what's about the flow in annulus? Or what's about the change of
pressure in annulus?

The flow will change the radial pressure forces. Is the neutral point changed
either?

\subsubsection{Perforating/Propellant}
During perforating w/o propellant, there is the quick change of
pressure. Meanwhile the fluid will be moved upward, which might make the
change of density of fluid.

\subsection{Temperature}
\begin{equation}
\Delta L_4 = L\beta\Delta t
\end{equation}
\begin{center}
\begin{tabular}{c l}
$\beta$ & coefficient of thermal expansion of the tubing material (for steel,
  it's $6.9E-6$ per Fahrenheit. \\
$\Delta t$ & change in average tubing temperature. \\
\end{tabular}
\end{center}

\subsection{Comments}
<Helical Buckling of Tubing Sealed in Packers>, was published in 1962. What's
the well operation at that time?

\subsubsection{Example operation}
The packer has been setted and tubing has been through the packer. After that
the real operation begin. The paper don't tell what's the operation
exactly. It just tell the pressure of tubing and annulus will be changed, the
fluid will be replaced, the temperature will be up or down. I don't think it
is the well completion.

But this information gives me some hints, which I should understand what the
operation have done during well completion. I mean whether the pressure, fluid
and temperature are changed.

\subsubsection{No Movement}
The first step is calculate the length of tubing that can move freely in
packer. Then there is a shoulder that restrains the downward movement of
tubing in packer. The paper says there is slack-off makes the tubing longer.

\subsubsection{The length of Tubing}

\section{Combination Tubing Strings Sealed in Packers}
In actual well, the uniform tubing string is not exist. The different size of
the cases and tubings that are different size combine together into the whole
strings. Meanwhile there are different fluids exist in the case and tubing.

The combination completion consists of one or more of the following: (1) more
than one size of tubing; (2) more than one size of casing, and (3) two or more
fluids in the tubing and/or annulus.

Based on the equations above, the sections of the tubings in the same size or
same fluid have to be calculated independantly and relatively.

\subsection{Hooke's Law}
Eq. \ref{eqHooke} is appled for each section that has the same size of tubing
and casing, and the same fluid. Be careful, the $F$ must be calculated for
each one.

\subsection{Helical Buckling}
Which Eq. \ref{eqBuckle1} or Eq. \ref{eqBuckle2} is appled is determined by
Eq. \ref{eqNeutral}.

The neutral point is 
\begin{equation}
n= \frac{F - \sum_{i=0}^n{L_iW_i}}{W_{n+1}} + \sum_{i=0}^nL_i
\end{equation}
The force $F$ in the above equation is fictitious.
\end{document}
